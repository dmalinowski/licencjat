\documentclass[licencjacka]{pracamgr}
\usepackage{polski}
\usepackage[utf8]{inputenc}
\usepackage{amssymb}
\usepackage{amsthm}
\usepackage{enumitem}
\usepackage{eufrak}
\usepackage{color}
%\usepackage[cp1250]{inputenc}

\author{Daniel Malinowski}
\nralbumu{292680}

\title{Metody dowodzenia prostoty grup}

\tytulang{Methods of proving the simplicity of groups}

\kierunek{Matematyka}

\opiekun{dra hab. Zbigniewa Marciniaka\\
  Instytut Matematyki\\}

\date{Czerwiec 2013}

\dziedzina{
11.1 Matematyka\\
}

%Klasyfikacja tematyczna według AMS (matematyka) lub ACM (informatyka)
\klasyfikacja{20. Group theory and generalizations}

\keywords{grupa prosta, grupa alternująca, specjalna rzutowa grupa
liniowa, lemat Iwasawy}

% Tu jest dobre miejsce na Twoje w?asne makra i~?rodowiska:
\newtheorem{deff}{Definicja}[section]
\newtheorem{thh}{Twierdzenie}[section]
\newtheorem{lemma}{Lemat}[section]
\newtheorem{fact}{Stwierdzenie}[section]
\newsymbol\eop 1003
% koniec definicji

\begin{document}
\maketitle

\begin{abstract}
  % TODO
    TODO
\end{abstract}

\tableofcontents
%\listoffigures
%\listoftables


\chapter*{Wprowadzenie}
\addcontentsline{toc}{chapter}{Wprowadzenie}

%TODO
TODO

\chapter{Wiadomości wstępne}

Rozdział ten zawiera przypomnienie pewnych definicji, własności
i~twierdzeń omawianych na podstawowym kursie
Algebry I oraz ustalenie oznaczeń.

\section{Oznaczenia}

W niniejszej pracy wielkimi literami alfabetu (np.
$G, H, K$) będą oznaczane grupy. Ich elementy będą oznaczanie małymi
literami alfabetu (np. $g, h, k$), przy czym przez $e$ będzie zawsze
oznaczany element neutralny. Rozważane grupy będą (w większości)
nieprzemienne, w~związku z~tym będzie stosowany zapis
multiplikatywny.

Jeżeli $H$ oraz $K$ są podzbiorami grupy $G$, to przez $HK$
będzie oznaczany podzbiór iloczynów $\{ h\cdot k \colon h \in H, k \in K\}\subseteq G$.

W związku z~tym oznaczeniem warto przytoczyć twierdzenie:

\begin{thh} \label{mult_groups} $ $ \\
    Jeżeli $H$ oraz $K$ są podgrupami grupy $G$, przy czym $K$ jest podgrupą normalną,
    to $HK$ jest podgrupą grupy $G$.
\end{thh}


\section{Grupy proste}

Przypomnijmy teraz podstawową definicję w~tej pracy.
\begin{deff}
    Nietrywialną grupę $G$ nazwiemy \emph{grupą prostą}, jeżeli nie ma ona podgrup normalnych różnych od $\{e\}$ oraz samej siebie.
\end{deff}
\begin{fact}
    Jedynymi (z dokładnością do izomorfizmu) przemiennymi grupami prostymi są skończone grupy cykliczne,
    których rząd jest liczbą pierwszą.
\end{fact}
Jest to prosta konsekwencja tego, że w~grupach przemiennych
wszystkie podgrupy są normalne oraz że każda inna grupa przemienna ma właściwą podgrupę cykliczną.
% jak zrobić odwołanie po szczegóły?


\section{Twierdzenia o~izomorfizmie}
Przejdźmy teraz do podstawowych twierdzeń o~izomorfizmie.
\begin{thh}[Pierwsze twierdzenie o~izomorfizmie \textcolor{red}{ -- odsyłacz}] $ $ \\
    Niech $ \varphi \colon G \to H$ będzie homomorfizmem grup. Oznaczmy $K = \ker{\varphi}$ oraz
    $H' = \mathrm{im}\varphi $. 
    Wówczas ma miejsce izomorfizm $$G/K \simeq H'.\quad\eop$$
\end{thh}
\begin{thh}[Drugie twierdzenie o~izomorfizmie \textcolor{red}{ -- odsyłacz}]$ $\\
    Niech $G$ będzie grupą,  $H_1, H_2$ jej podgrupami normalnymi,
    przy czym $H_2 \leq H_1$. 
    Wówczas $H_2 \trianglelefteq H_1$, $H_1/H_2 \trianglelefteq G/H_2$ i~ma miejsce izomorfizm
    $$ (G/H_2) / (H_1/H_2) \simeq G/H_1.\quad\eop$$
\end{thh}
\begin{thh}[Trzecie twierdzenie o~izomorfizmie \textcolor{red}{ -- odsyłacz}] $ $ \\
    Niech $G$ będzie grupą, $H$ oraz $H_1$
    -- jej podgrupami, przy czym $H_1$ jest podgrupą normalną w~$G$. 
    Wówczas $H\cap H_1$ jest podgrupą normalną w~$H$ oraz ma miejsce izomorfizm
    $$ H/(H\cap H_1) \simeq HH_1 / H_1.\quad\eop$$
\end{thh}


\section{Komutant i~abelianizacja}
Poniżej przedstawionych jest kilka użytecznych wiadomości
o~komutancie.
\begin{deff}
    Niech $G$ będzie dowolną grupą. Wówczas \emph{komutantem grupy $G$} nazywamy podgrupę $G$ generowaną przez wszystkie elementy postaci
    $aba^{-1}b^{-1}$, gdzie $a, b \in G$. Komutant grupy $G$ oznaczamy przez $[G, G]$.
\end{deff}
\begin{thh}[O komutancie \textcolor{red}{ -- odsyłacz}] $ $\\
    Komutant $[G,G]$ jest podgrupą normalną $G$, przy czym grupa ilorazowa $G/[G,G]$ jest grupą abelową.
    Ponadto dla dowolnej podgrupy normalnej $H \trianglelefteq G$ takiej, że $G/H$ jest abelowa, zachodzi $[G,G] \le H$
    \quad$\eop$.
\end{thh}
\begin{deff}
    Przekształcenie kanoniczne $G \to G/[G,G]$ (rzutowanie na grupę ilorazową) nazywamy
    homomorfizmem abelianizacji, zaś grupę ilorazową $G/[G,G]$ -- abelianizacją grupy $G$.
\end{deff}

W skrajnym przypadku abelianizacja grupy jest trywialna, co prowadzi do ważnego pojęcia
grupy doskonałej:

\begin{deff}
    \emph{Grupą doskonałą} nazwiemy dowolną grupę, która jest równa swojemu komutantowi.
\end{deff}

Grupami doskonałymi zajmiemy się w~dalszej części pracy -- przy
lemacie Iwasawy. Na razie zanotujmy prosty fakt:

\begin{fact}
    Nieprzemienne grupy proste są grupami doskonałymi.\quad$\eop$
\end{fact}


\section{Działanie grupy na zbiorze}
Na koniec tego rozdziału przyjrzyjmy się użytecznej
własności grup -- możliwości działania na zbiorach.

\begin{deff}
    Niech $G$ będzie grupą, a~$X$ -- zbiorem. Mówimy, że \emph{$\rho$ jest działaniem grupy $G$ na zbiorze $X$},
    jeżeli każdemu elementowi $g \in G$ przyporządkowane jest przekształcenie $\rho_g\colon X \to X$, takie, że:
    \begin{itemize}
        \item $\rho_e = \mathrm{id}_X$,
        \item $\rho_g \circ \rho_h = \rho_{gh}$, dla dowolnych $g, h \in G$.
    \end{itemize}
    Jeżeli sposób działania ($\rho$) wynika z~kontekstu, to zamiast $\rho_g(x)$ będziemy pisać $x^g$.
\end{deff}

Zgrabniejszy opis działania grupy na zbiorze daje następne
twierdzenie. Zanim jednak do niego przejdziemy, przypomnijmy
jeszcze jedną definicję.

\begin{deff}
    Niech $X$ będzie dowolnym zbiorem. Wówczas \emph{grupą symetrii zbioru X} nazywamy zbiór bijekcji $X \to X$,
    wraz z~operacją składania. Grupę tę oznaczamy $S_X$.
\end{deff}

\begin{thh}[O działaniu grupy na zbiorze] $ $ \\
    Niech $G$ będzie grupą, a~$X$ -- zbiorem. Wówczas $\rho$ jest działaniem $G$ na $X$ wtedy i~tylko wtedy,
    gdy $\rho$ jest homomorfizmem z~$G$ w~grupę symetrii zbioru $X$.\quad$\eop$
\end{thh}

Z działaniem grupy na zbiorze związane jest dużo ważnych definicji
i~twierdzeń. Poniżej przytoczone są te najistotniejsze z~punktu
widzenia tej pracy.

\begin{deff}
    Załóżmy, że $\rho$ jest działaniem grupy $G$ na zbiorze $X$ oraz $x \in X$. Wówczas:
    \begin{enumerate}[label=\alph*)]
     \item \emph{Stabilizatorem punktu $x$ (grupą izotropii $x$)} nazwiemy zbiór elementów $\{g \in G\colon x^g = x \}$.
                    Stabilizator punktu $x$ oznaczamy $G_x$.
     \item \emph{Orbitą punktu $x$} nazwiemy podzbiór $X$ równy $\{y \in X \colon \exists_{g \in G} x^g = y \}$.
                    Orbitę punktu $x$ oznaczamy $G(x)$.
    \end{enumerate}
% punkt stały ?
\end{deff}

Podstawowe własności tych obiektów przedstawia następujące stwierdzenie:
\begin{fact} [\textcolor{red}{odsyłacz}]
    Załóżmy, że $\rho$ jest działaniem grupy $G$ na zbiorze $X$ oraz $x, y \in X$. Wówczas:
    \begin{enumerate}[label=\alph*)]
     \item $G_x$ jest podgrupą $G$.
     \item $G(x)$ i~$G(y)$ są równe lub rozłączne (orbity tworzą rozbicie zbioru $X$).\quad$\eop$
    \end{enumerate}
\end{fact}

Zanim przejdziemy do ważniejszych twierdzeń opisujących orbity
i~stabilizatory, przypomnijmy wcześniej, jakie własności może mieć
działanie grupy na zbiorze.

\begin{deff}
    Załóżmy, że $\rho$ jest działaniem grupy $G$ na zbiorze $X$.
    \begin{enumerate}[label=\alph*)]
     \item \emph{$\rho$ jest działaniem tranzytywnym (przechodnim)}, jeżeli wszystkie elementy $X$ tworzą jedną orbitę.
     \item \emph{$\rho$ jest działaniem wiernym}, jeżeli $\rho$ jest iniekcją jako homomorfizm $G \to S_X$.
     \item \emph{$\rho$ jest działaniem nietrywialnym}, jeżeli $\rho$ nie jest homomorfizmem stałym $G \to S_X$.
    \end{enumerate}
\end{deff}

Jak to zostało wcześniej zapowiedziane, na koniec przytoczmy kilka
ważnych twierdzeń pokazujących zależność między orbitami
a~stabilizatorami.

\begin{thh}\label{conj_stab} $ $[\textcolor{red}{odsyłacz}]\\
    Załóżmy, że $\rho$ jest działaniem grupy $G$ na zbiorze $X$ oraz $x, y \in X$ należą do jednej orbity.
    Wówczas grupy $G_x$ oraz $G_y$ są sprzężone w grupie $G$.
\end{thh}

\begin{thh}[O orbitach i~stabilizatorach] [\textcolor{red}{odsyłacz}]$ $\\
    Załóżmy, że $\rho$ jest działaniem grupy $G$ na zbiorze $X$ oraz $x\in X$.
    Wówczas $|G(x)| = [G : G_x]$.
\end{thh}

\begin{thh}[Równanie klas] [\textcolor{red}{odsyłacz}] $ $\\
    Przy założeniach z~poprzedniego twierdzenia zachodzi
    $$ |X| = \sum_{i=1}^k [G : G_{x_i}] ,$$
    gdzie $x_1, x_2, \cdots, x_k$ są reprezentantami wszystkich orbit działania $\rho$.
\end{thh}



\chapter{Prostota grupy alternującej $A_n$}

Zanim udowodnimy główną tezę tego rozdziału, czyli twierdzenie, że $A_n$
jest grupą prostą dla $n \ge 5$, przypomnimy znane własności o tej
grupie oraz udowodnimy kilka mniej znanych.

\section{Przypomnienie wiadomości o $S_n$ oraz $A_n$}

W poprzednim rozdziale wprowadziliśmy definicję grupy $S_X$ symetrii
zbioru $X$. Ważnym przypadkiem szczególnym jest sytuacja, gdy $X$
jest zbiorem skończonym o $n$ elementach. Wówczas, jako że grupy
symetrii zbiorów równolicznych są izomorficzne, grupę $S_X$ będziemy
oznaczać $S_n$ i bez straty ogólności przyjmiemy, że jej elementami
są permutacje zbioru $\{1, 2, \cdots, n\}$.
%napisać coś o składaniu permutacji, tzn. że inaczej niż funkcji?

\begin{fact}
    Rząd grupy $S_n$ wynosi $n!$.\quad$\eop$
\end{fact}

Ważnym sposobem przedstawienia elementów grupy $S_n$ jest rozkład na
cykle.

\begin{deff}    %co z~cyklami jednoelementowymi?
    Permutację $\sigma \in S_n$ nazwiemy \emph{cyklem długości $k$},
    jeżeli istnieją różne elementy $c_1, c_2, \ldots, c_k \in \{1, 2, \cdots, n\}$ takie, że
    $$ \sigma(x) = \left\{
                \begin{array}{ll}
                    c_{i+1}, & \textrm{jeżeli $x = c_i$}\\
                    c_1,     & \textrm{jeżeli $x = c_k$}\\
                    x,       & \textrm{w przeciwnym przypadku}
                \end{array} \right.
    $$
\end{deff}

Cykle zapisujemy w postaci $(c_1, c_2, \cdots, c_k)$. Oczywiście zapis cyklu nie jest jednoznaczny; 
następujące zapisy: $(c_1, c_2, \cdots, c_k) = (c_k, c_1, c_2, \cdots, c_{k-1}) = (c_2, c_3, \cdots, c_k,
c_1)$ reprezentują ten sam cykl.

Dla $\sigma \in S_n$ oraz $x\in\{1, 2,\cdots, n\}$ zbiór
$\{x,\sigma(x),\sigma^2(x),\ldots\}\subseteq\{1,\ldots,n\}$ jest skończony, zatem istnieją
liczby $k<l\leq n$ takie, że $\sigma^k(x)=\sigma^l(x)$, a stąd $\sigma^{l-k}(x)=x$. Jeśli $d$ jest najmniejszą
liczbą całkowitą dodatnią taką, że $\sigma^d(x)=x$, to mamy cykl $\left(x,\sigma(x),\ldots,\sigma^{d-1}(x)\right)$.
Powtarzając tę procedurę z~niewybranymi jeszcze elementami $x$, dostaniemy rozkład na cykle:

\begin{thh} $ $ \\      %co z~cyklami jednoelementowymi?
    Każdą permutację $\sigma \in S_n$ można przedstawić jako iloczyn rozłącznych cykli,
    czyli takich $(c_1, c_2, \cdots, c_k)$, $(d_1, d_2, \cdots, d_l)$, że $\{ c_1, c_2, \cdots, c_k \} \cap \{ d_1, d_2, \cdots, d_l \} = \emptyset$
    przy czym każdy element ze zbioru $\{1, 2, \cdots, n\}$ znajduje się w pewnym cyklu.
    Przedstawienie jest jednoznaczne z~dokładnością do kolejności cykli.
\end{thh}

\pagebreak[2]

Przejdźmy teraz do zdefiniowania podgrupy $A_n$ grupy $S_n$. Załóżmy
do końca tego rozdziału, że $n \ge 2$.

\begin{deff}
    \emph{Transpozycją} nazwiemy dowolny cykl długości 2.
\end{deff}

Transpozycję są cegiełkami, z~których można budować permutacje, tzn.

\begin{thh} $ $ \\
    Każda permutacja jest iloczynem pewnej liczby transpozycji.
\end{thh}

Rozkład permutacji na transpozycje nie musi być jednoznaczny. Np.
$(1, 2) (2, 4) (4, 2) = (1, 2)$ oraz $(1, 2) (2, 3) (3, 4) (4, 1) =
(4, 2) (2, 3)$. Jednoznaczna natomiast jest parzystość liczby
transpozycji w rozkładzie.

\begin{deff}
    Permutację, którą można przedstawić w postaci iloczynu parzystej liczby
    transpozycji, nazwiemy \emph{permutacją parzystą},
    w przeciwnym przypadku -- \emph{nieparzystą}.
    Podgrupę wszystkich permutacji parzystych grupy $S_n$ nazywamy \emph{grupą alternującą} i oznaczamy $A_n$.
\end{deff}


Poprawność definicji wynika z~twierdzenia:

\begin{thh}\label{thm_A_n} $ $[\textcolor{red}{odsyłacz}] \\
    Parzystość liczby transpozycji w rozkładzie permutacji na transpozycje nie zależy od rozkładu.
    Permutacje o parzystej liczbie transpozycji tworzą podgrupę normalną grupy $S_n$ indeksu 2, czyli
    rzędu $n!/2$.
\end{thh}

Warto tu jeszcze wspomnieć o tym, które cykle są permutacjami
parzystymi, a które nie. Mianowicie, trochę wbrew swojej nazwie,
cykle o długości nieparzystej są parzyste, a~o~długości parzystej --
nieparzyste. Stąd prawdziwe jest:

\begin{fact}\label{fact_typ_A_n}
    Permutacja $\sigma \in S_n$ jest parzysta wtedy i tylko wtedy, kiedy w rozkładzie na cykle
    zawiera parzystą liczbę cykli o parzystej długości.\quad$\eop$
\end{fact}


\section{Klasy sprzężoności $S_n$ i $A_n$}

W celu udowodnienia prostoty grupy $A_n$ zbadamy klasy
sprzężoności tej grupy. Najpierw zajmiemy się jednak prostszym
problemem -- klasami sprzężoności $S_n$.

\begin{deff}
    \emph{Typem cyklowym} permutacji $\sigma \in S_n$ nazwiemy listę długości cykli występujących w $\sigma$,
    tzn. ciąg $(1^{i_1}, 2^{i_2}, \cdots, n^{i_n})$,
    gdzie $i_k$ to liczba cykli długości $k$ w rozkładzie $\sigma$ na cykle rozłączne.
\end{deff}
W celu uproszczenia zapisu można omijać długości cykli, które nie
występują w rozkładzie. Dla przykładu typem cyklowym transpozycji
jest $(1^{n-2}, 2^1)$, a identyczności -- $(1^n)$.

Okazuje się, że w grupie $S_n$ typ cyklowy jednoznacznie wskazuje na
klasę sprzężoności:

\begin{thh} $ $ \\
    Permutacje $\pi, \sigma \in S_n$ są sprzężone wtedy i tylko wtedy, gdy ich indeks cyklowy jest taki sam.
\end{thh}
\begin{proof}

    [\textcolor{red}{\bf Wygląda na to, że stosuje Pan funkcje do argumentu od strony lewej do prawej,
    czyli działa grupą permutacji z prawej strony. Warto to chyba zapowiedzieć zaraz po definicji permutacji}]

    Niech $\lambda = (c_1, c_2, \cdots, c_k)$ będzie cyklem w $S_n$, $c_{k+1} = c_1$ oraz $\gamma \in S_n$.
    Wówczas
    $(\gamma^{-1} \lambda \gamma)(\gamma(c_i)) = (\lambda \gamma)(c_i) = \gamma(c_{i+1})$,
    na pozostałych elementach $\gamma^{-1} \lambda \gamma$ jest stałe,
    zatem $\gamma^{-1} (c_1, c_2, \cdots, c_k) \gamma = (\gamma(c_1), \gamma(c_2), \cdots, \gamma(c_k))$.
    Stąd również

        $$\gamma \Big(c_1^1, c_2^1, \cdots, c_{k_1}^1\Big) \cdots \Big(c_1^m, c_2^m, \cdots, c_{k_m}^m \Big) \gamma^{-1} = $$
        $$ = \gamma \Big(c_1^1, c_2^1, \cdots, c_{k_1}^1\Big) \gamma^{-1} \gamma  \cdots \gamma ^{-1} \gamma \Big(c_1^m, c_2^m, \cdots, c_{k_m}^m\Big) \gamma^{-1} = $$
        $$ = \Big(\gamma (c_1^1), \gamma (c_2^1), \cdots, \gamma (c_{k_1}^1)\Big) \cdots \Big(\gamma (c_1^m), \gamma (c_2^m), \cdots, \gamma (c_{k_m}^m)\Big)$$

    Jeżeli cykle $\Big(c_1^1, c_2^1, \cdots, c_{k_1}^1\Big) \cdots \Big(c_1^m, c_2^m, \cdots, c_{k_m}^m \Big)$ były rozłączne,
    to również powstałe po sprzężeniu cykle są rozłączne.
    Jest ich tyle samo i mają te same długości, zatem rzeczywiście sprzężenie zachowuje typ permutacji.

    Wystarczy jeszcze pokazać, że permutacje o tym samym typie są sprzężone.
    Niech $\pi =    \Big(c_1^1, c_2^1, \cdots, c_{k_1}^1\Big) \cdots \Big(c_1^m, c_2^m, \cdots, c_{k_m}^m \Big)$
    oraz  $\sigma = \Big(d_1^1, d_2^1, \cdots, d_{k_1}^1\Big) \cdots \Big(d_1^m, d_2^m, \cdots, d_{k_m}^m \Big)$.
    Wówczas permutacja $\gamma \colon c_i^j \mapsto d_i^j$ jest taka, że $\gamma \pi \gamma^{-1} = \sigma$.
\end{proof}

Klasy sprzężoności permutacji parzystych w $S_n$ mogą rozpaść się
na kilka mniejszych w~$A_n$, gdyż $A_n \le S_n$, czyli w $A_n$ jest
mniejszy wybór elementów, którymi możemy sprzęgać. Okazuje się, że
rzeczywiście niektóre z~tych klas rozpadają się na dwie:

\begin{thh}\label{thm_o_klasach_A_n} $ $ \\
    Typy cyklowe permutacji parzystych, które zawierają cykl o~parzystej długości
    lub dwa cykle o~tej samej nieparzystej długości (możliwe, że o długości 1)
    odpowiadają jednej klasie sprzężoności w $A_n$.
    Pozostałe typy permutacji parzystych odpowiadają dwóm równolicznym klasom sprzężoności w $A_n$.
\end{thh}
\begin{proof}
    Zauważmy najpierw, że jeżeli $\sigma \in A_n$ jest centralizowane przez pewną nieparzystą permutację $\gamma$
    (tzn. $\sigma = \gamma \sigma \gamma ^{-1}$),
    to $\sigma$ jest sprzężona w $A_n$ ze wszystkimi permutacjami o tym samym typie cyklowym.
    Jest tak dlatego, że z każdą taką permutacją $\psi$ permutacja $\sigma$ jest sprzężona w $S_n$ przez pewną permutację $\pi$,
    tzn. $\psi = \pi \sigma \pi^{-1}$.
    Ale również $\psi = \pi \gamma \sigma \gamma^{-1} \pi^{-1} = (\pi \gamma) \sigma (\pi \gamma)^{-1}$.
    Jedna z permutacji $\pi$ lub $\pi \gamma$ jest parzysta, więc rzeczywiście $\psi$ oraz $\sigma$ są sprzężone w $A_n$.

    Jeżeli $\sigma$ ma w rozkładzie na cykle rozłączne cykl parzystej długości $\lambda$,
    to jest przez niego centralizowana (a jest on nieparzysty),
    a jeżeli ma dwa cykle o tej samej nieparzystej długości $(c_1, c_2, \cdots, c_m)$ oraz $(d_1, d_2, \cdots, d_m)$,
    to jest centralizowana przez nieparzystą permutację $(c_1, d_1)(c_2, d_2) \cdots (c_m, d_m)$,
    czyli rzeczywiście $\sigma$ jest sprzężona ze wszystkimi elementami o~tym samym typie cyklowym.

    W przypadku, gdy $\sigma$ nie jest centralizowana przez żadną nieparzystą permutację,
    to permutacje o tym samym typie cyklowym co $\sigma$ rozpadają się na dwie klasy sprzężoności --
    $\{\lambda \sigma \lambda^{-1} \colon \lambda \in S_n \backslash A_n \}$ oraz
    $\{\pi \sigma \pi^{-1} \colon \pi \in  A_n \}$.
    Są one równoliczne, gdyż są sprzężone w $S_n$.

    Z takim przypadkiem mamy do czynienia, gdy $\sigma$ w rozkładzie na cykle rozłączne
    ma tylko cykle o różnych nieparzystych długościach.
    Przy centralizowaniu każdy taki cykl musi przejść na cykl o tej samej długości, czyli na siebie.
    Ponadto pierwsze elementy z cykli muszą przejść na elementy ze swoich cykli,
    a obraz pozostałych elementów jest już przez to wyznaczony jednoznacznie.
    W związku z tym $\sigma$ może być centralizowane tylko przez permutacje,
    które są równe iloczynowi potęg cykli z rozkładu $\sigma$,
    czyli tylko przez permutacje parzyste.
\end{proof}

Zanim udowodnimy prostotę grup $A_n$ pokażemy jeszcze dwa przydatne
lematy.

\begin{lemma}\label{lemma_big_con}
    Dla $n \ge 5$ klasy sprzężoności elementów nietrywialnych $A_n$ mają co najmniej $n$ elementów.
\end{lemma}
\begin{proof}
    Niech $\sigma \in A_n$, $\sigma \ne \mathrm{id}$.
    Oszacujmy ile permutacji ma ten sam typ cyklowy co $\sigma$.

    Jeżeli $\sigma$ zawiera cykl długości $\ge 3$, to samych permutacji o tym samym typie co $\sigma$, które w tym cyklu mają liczbę 1 jest co najmniej $(n-1)(n-2)$,
    gdyż możemy wybrać na co przechodzi liczba 1 i~na co przechodzi wybrana liczba.
    \textcolor{red}{$\leftarrow$ \bf niejasne}
    Stąd z poprzedniego twierdzenia w~klasie sprzężoności $\sigma$ jest co najmniej $\frac{(n-1)(n-2)}{2} \ge n$ elementów (bo $n \ge 5$).

    W przeciwnym przypadku $\sigma$ zawiera co najmniej dwa cykle długości 2.
    Analogicznie dostajemy, że samych permutacji o tym samym typie co $\sigma$,
    które w jednym z tych cykli mają liczbę 1, a w drugim liczbę 2 jest co najmniej $(n-2)(n-3)$,
    więc z poprzedniego twierdzenia w tym przypadku również rozmiar klasy sprzężoności $\sigma$ wynosi co najmniej $(n-2)(n-3) \ge n$.
\end{proof}



\begin{lemma}\label{lemma_3cycles}
    Cykle długości 3 generują całą grupę $A_n$.
\end{lemma}
\begin{proof}
    Każdą permutację  $\sigma \in A_n$ można przedstawić w postaci iloczynu parzystej liczby transpozycji
    $\sigma = \lambda_1 \lambda_2 \cdots \lambda_{2m-1} \lambda_{2m} = (\lambda_1 \lambda_2) \cdots (\lambda_{2m-1} \lambda_{2m})$.
    Stąd wystarczy przedstawić iloczyn dwóch transpozycji $\lambda_1 \lambda_2$ jako iloczyn cykli długości 3, a dostaniemy tezę.

    Jeżeli $\lambda_1 = \lambda_2$, to $\lambda_1 \lambda_2 = \mathrm{id}$.
    Gdy $\lambda_1$ i $\lambda_2$ są rozłączne, czyli $\lambda_1 = (a, b)$, $\lambda_2 = (c, d)$, to $\lambda_1 \lambda_2 = (a, c, d)(a, c, b)$.
    Jeżeli natomiast $\lambda_1$ i $\lambda_2$ mają jeden element wspólny,
    czyli $\lambda_1 = (a, b)$, $\lambda_2 = (a, c)$, to $\lambda_1 \lambda_2 = (a, b, c)$.
\end{proof}


\section{Prostota $A_n$}
Jesteśmy już gotowi, żeby udowodnić twierdzenie:

\begin{thh}[O prostocie $A_n$] $ $ \\
    Grupa alternująca $A_n$ jest prosta dla $n \ge 5$.
\end{thh}
\begin{proof}
    Dowód przeprowadzimy przez indukcję ze względu na $n$.

    Pokażemy najpierw, że grupa $A_5$ jest prosta.

    Na podstawie Stwierdzenia \ref{fact_typ_A_n} wiemy, że elementy $A_5$ mają jeden z~następujących typów cyklowych:
    $(1^5), (1^2, 3^1), (1^1, 2^2)$ lub $(5^1)$.
    z twierdzenia \ref{thm_o_klasach_A_n} każdy z~pierwszych czterech odpowiada jednej klasie sprzężoności, a~ostatni dwóm -- równolicznym.
    Stąd klasy sprzężoności $A_5$ mają rozmiary: 1, 20, 15, 12, 12.

    Załóżmy nie wprost, że $H$ jest nietrywialną, właściwą podgrupą normalną $A_5$.
    Wówczas $H$ musi być sumą pewnych klas sprzężoności $A_5$, w tym klasy sprzężoności elementu neutralnego.
    Ponadto rozmiar $H$ musi być dzielnikiem rozmiaru $A_5 = 60$.
    Najmniejszy nietrywialny możliwy rozmiar sumy klas sprzężoności wraz z~trywialną wynosi 13.
    Stąd $|H| = 15$, $|H| = 20$ lub $|H| = 30$.
    Ale żaden podzbiór multizbioru $\{1, 12, 12, 15, 20 \}$ zawierający jedynkę nie sumuje się do potencjalnego rozmiaru $H$,
    zatem takie $H$ nie może istnieć -- $A_5$ jest grupą prostą.

    Załóżmy zatem, że $n \ge 6$ oraz grupa $A_{n-1}$ jest prosta.
    Pokażemy, że $A_n$ również jest prosta.

    Załóżmy nie wprost, że $H$ jest nietrywialną, właściwą podgrupą normalną w $A_n$.

    Jeżeli $H$ zawiera pewną nietrywialną permutację $\sigma$, która ma punkt stały $a \in \{ 1, 2, \cdots, n \}$, to niech $K = H_a$.
    Wówczas $K \simeq A_{n-1}$ oraz (np. z~trzeciego twierdzenia o izomorfizmie) $H \cap K$ jest podgrupą normalną w $K$.
    Ale $\sigma \in H \cap K$ oraz $K$ jest grupą prostą, zatem z~założenia indukcyjnego $H \cap K = K$.
    Stąd $H$ zawiera pewien element o typie $(1^{n-3}, 3^1)$, a~zatem wszystkie elementy
    tego typu, na mocy twierdzenia \ref{thm_o_klasach_A_n}, gdyż $n-3 \ge 3$.
    Ale z~lematu \ref{lemma_3cycles}    cykle o długości 3 generują całe $A_n$, stąd $H = A_n$ --
    sprzeczność z~założeniem, że $H$ jest podgrupą właściwą.

    Jeżeli natomiast żaden nietrywialny element $H$ nie ma punktu stałego, to $|H| \le n$.
    W~przeciwnym przypadku istniałyby dwie różne permutacje $\pi, \sigma \in H$ takie, że $\pi(1) = \sigma(1)$.
    Wtedy $\gamma = \pi \sigma ^{-1} \ne \mathrm{id}$, $\gamma \in H$ oraz $\gamma(1) = 1$ -- sprzeczność.
    Stąd rzeczywiście $|H| \le n$.
    Ale z~lematu \ref{lemma_big_con} $H$ jako nietrywialna suma pewnej liczby klas sprzężoności w tym trywialnej musiałaby mieć rozmiar co najmniej $n+1$.
    Stąd w tym przypadku również otrzymujemy sprzeczność.

    We wszystkich przypadkach otrzymaliśmy sprzeczność, czyli rzeczywiście $A_n$ jest grupą prostą,
    czyli z~indukcji $A_n$ jest grupą prostą dla wszystkich $n \ge 5$.
\end{proof}

Można się jeszcze zastanawiać, jak wygląda $A_n$ dla $n < 5$. Z
twierdzenia \ref{thm_A_n} wiemy, że $|A_n| = n!/2$. Zatem $A_2$ jest
grupą trywialną. $A_3$ ma
3 elementy -- jest grupą cykliczną o 3 elementach, więc jest prosta.
Natomiast grupa $A_4$ nie jest prosta -- jej czteroelementowa
podgrupa $H = \langle (1,2)(3,4), (1,3)(2,4) \rangle$ jest normalna,
gdyż składa się ze wszystkich elementów o rzędzie $\le 2$.



\chapter{Lemat Iwasawy}
W tym rozdziale przedstawione zostanie jedno z~ważniejszych narzędzi
do dowodzenia prostoty grup -- lemat Iwasawy. Lecz najpierw
wprowadzimy nowe pojęcie --  prymitywność.

\section{Prymitywne działanie grupy}
Jak zostało to już wspomniane w~wiadomościach wstępnych, działanie
grupy $G$ na zbiorze $X$ jest tranzytywne, jeżeli elementy $X$
tworzą jedną orbitę, czyli dla dowolnych $x, y \in X$ istnieje $g
\in G$ takie, że $x^g = y$. Teraz uogólnimy to pojęcie.

\begin{deff}
    Załóżmy, że $\rho$ jest działaniem grupy $G$ na zbiorze $X$. \\
    Powiemy, że $\rho$ jest \emph{działaniem $k$-tranzytywnym ($k$-przechodnim)},
    jeżeli dla dowolnych ciągów $k$ elementowych $(a_1, a_2, \cdots, a_k)$ oraz $(b_1, b_2, \cdots, b_k)$,
    które składają się z~różnych elementów z~$X$, istnieje taki element $g$ z~grupy $G$, że
    $a_i^g = b_i^g$ dla każdego $i = 1, 2, \cdots, k$.
\end{deff}
W szczególności $1$-tranzytywność to jest dokładnie to samo, co
zwykła tranzytywność.

Aby zilustrować to pojęcie, policzmy jaki jest stopień tranzytywności naturalnego działania
$S_n$ oraz $A_n$ na zbiorze $X = \{1, 2, \cdots,
n\}$, tzn. takiego, w~którym $i^\sigma = \sigma(i)$.

Jak łatwo zauważyć, działanie $S_n$ jest $n$-tranzytywne -- skoro
$S_n$ składa się ze wszystkich permutacji, to zawsze możemy
odwzorować ciąg $(a_1, a_2, \cdots, a_n)$ na $(b_1, b_2, \cdots,
b_n)$, gdyż jak założyliśmy w~definicji, wszystkie $a_i$ jak
i~wszystkie $b_i$ są parami różne. Stąd również działanie $S_n$ jest
$k$-tranzytywne dla każdego $k \le n$.

Natomiast w~$A_n$ nie ma wszystkich permutacji, zatem działanie
$A_n$ nie może być $n$-tranzytywne. Nie może być również
$(n-1)$-tranzytywne, gdyż skoro ustalimy na co przejdzie pierwsze $n-1$
elementów $X$ i~ma to być permutacja, to obraz ostatniego elementu
też jest ustalony, czyli wybór $(n-1)$ pozycji jest tak na prawdę
wyborem wszystkich $n$ pozycji, a na wszystkich elementach nie
możemy dowolnie ustalić permutacji. Zauważmy jednak, że działanie
$A_n$ jest $(n-2)$-tranzytywne. Rzeczywiście, chcąc żeby $a_i$
przeszło na $b_i$ dla $i = 1, 2, \cdots, (n-2)$ mamy do wyboru dwie
permutacje (z $S_n$). Jedna z~nich odwzorowuje $x \mapsto y, x'
\mapsto y'$, a~druga $x \mapsto y', x' \mapsto y$, gdzie $x, x'$ to
elementy różne od wszystkich  $a_i$, a~$y, y'$ to elementy różne od wszystkich
$b_i$. Ale te permutacje różnią się o~transpozycję $(y, y')$, zatem
jedna z~nich jest parzysta, czyli należy do $A_n$, więc rzeczywiście
możemy odwzorować $(a_1, a_2, \cdots, a_{n-2})$ na $(b_1, b_2,
\cdots, b_{n-2})$. Stąd działanie $S_n$ jest $k$-tranzytywne dla
każdego $k \le n-2$.

Wprowadzimy teraz własność prymitywności. Jest to własność pomiędzy
tranzytywnością a~$2$-tranzytywnością.

\begin{deff}
    Załóżmy, że $\rho$ jest działaniem grupy $G$ na zbiorze $X$. \\
    \emph{Systemem bloków} działania $\rho$ nazywamy podział zbioru $X$ zachowywany przez $\rho$,
    tzn. rodzinę zbiorów $\mathfrak{A} = \{Y_i \colon i \in I \}$, które są niepuste, parami rozłączne, sumują się do $X$
    oraz dla dowolnych $Y \in \mathfrak{A}, x, x' \in Y$ oraz $g \in G$
    oba elementy $x^g$ oraz $x'^g$ znajdują się razem w~jednym zbiorze $Y' \in \mathfrak{A}$.
\end{deff}

Zauważmy, że zawsze mamy co najmniej dwa systemy bloków -- jeden
blok z~całym zbiorem $\mathfrak{A} = \{X\}$ oraz system z~wszystkimi
blokami jednoelementowymi $\mathfrak{A} = \{\{x\} \colon x \in X\}$.
W~związku z~tym naturalna jest definicja:

\begin{deff}
    \emph{Nietrywialnym systemem bloków} nazywamy dowolny system bloków,
    który jest różny od dwóch wyżej wspomnianych -- z~jednym blokiem lub z~blokami jednoelementowymi.
\end{deff}

Teraz jesteśmy już gotowi na wprowadzenie pojęcia prymitywności.

\begin{deff}
    Załóżmy, że $\rho$ jest działaniem grupy $G$ na zbiorze $X$. \\
    Działanie \emph{$\rho$ nazywamy prymitywnym}, jeśli nie istnieje nietrywialny system bloków działania $\rho$.
\end{deff}

Aby lepiej zrozumieć tą własność, pokażemy, że rzeczywiście jest to
własność pomiędzy tranzytywnością oraz $2$-tranzytywnością.

\begin{thh}
    Załóżmy, że $\rho$ jest nietrywialnym działaniem grupy $G$ na zbiorze $X$. Wówczas:
    \begin{enumerate}[label=\alph*)]
     \item Jeżeli $\rho$ jest prymitywne, to jest tranzytywne.
     \item Jeżeli $\rho$ jest $2$-tranzytywne, to jest prymitywne.
    \end{enumerate}
\end{thh}

\begin{proof}% jak to zrobić, żeby było ładne?
%$ $\\
    \begin{enumerate}[label=Ad \alph*)]
     \item  Załóżmy nie wprost, że $\rho$ nie jest tranzytywne.
                    Wówczas rozbicie $X$ na orbity daje nietrywialny system bloków.
                    Rzeczywiście, z~nieprzechodniości dostajemy, że liczba bloków wynosi co najmniej 2 a
                    z nietrywialności $\rho$ -- któryś blok ma co najmniej 2 elementy.
                    Ostatecznie $\rho$ permutuje elementy orbit, więc w szczególności je zachowuje.
                    Znaleźliśmy nietrywialny system bloków działania $\rho$, czyli sprzeczność -- $\rho$ nie jest prymitywne.
                    Stąd $\rho$ musi być tranzytywne.
     \item  Załóżmy nie wprost, że $\rho$ nie jest prymitywne.
                    Wówczas istnieje nietrywialny system bloków $\mathfrak{A} = \{Y_i \colon i \in I \}$,
                    w którym istnieją $Y_1, Y_2 \in \mathfrak{A}$ takie, że $|Y_1| > 1$.
                    Niech więc $x,y \in Y_1, z \in Y_2$ gdzie $x \ne y$.
                    Z 2-tranzytywności możemy odwzorować parę $(x, y)$ na parę $(x, z)$, co daje sprzeczność z~definicją systemu bloków.
                    Stąd $\rho$ musi być prymitywne.
    \end{enumerate}
\end{proof}

Oczywiście możliwe jest, że grupa działa tranzytywnie a nie
prymitywnie, lub prymitywnie, a nie 2-tranzytywnie.

Jako pierwszy przykład możemy rozważyć naturalne działanie
czteroelementowej grupy $H= \langle (1,2)(3,4), (1,3)(2,4) \rangle$
będącej podgrupą $S_4$ na zbiorze 4 elementowym. Jak łatwo widać
jest ono przechodnie. Nie jest jednak prymitywne, gdyż zachowuje ono
np. system bloków $\{\{1,2\}, \{3,4\}\}$.

Jako drugi przykład rozważmy działanie $A_3$ na zbiorze $\{1, 2,
3\}$. Jak pokazaliśmy wcześniej nie jest ono 2-tranzytywne, ale jest
tranzytywne. To, że jest to również działanie prymitywne wynika
z~następującego lematu:

\begin{lemma}
    Załóżmy, że $\rho$ jest tranzytywnym działaniem grupy $G$ na zbiorze $X$.
    Wówczas w dowolnym systemie bloków wszystkie bloki są równych rozmiarów.
\end{lemma}
\begin{proof}
Rzeczywiście, jeżeli $Y_1, Y_2$ są blokami, to skoro możemy
odwzorować $y_1 \in Y_1$ na $y_2 \in Y_2$, To całe $Y_1$ musi być
przekształcone w $Y_2$ (z własności systemu bloków), stąd $|Y_1| \le
|Y_2|$. Analogicznie $|Y_2| \le |Y_1|$, zatem $|Y_1| = |Y_2|$.
\end{proof}


W tym przypadku bloki w nietrywialnym systemie bloków muszą mieć
rozmiary 1 i 2, czyli różne, więc nietrywialny system bloków nie
może istnieć.

Udowodnijmy teraz jeszcze jedno stwierdzenie, które jest użyteczne w
dowodzie lematu Iwasawy.

\begin{lemma}\label{max_group}
    Załóżmy, że $\rho$ jest tranzytywnym działaniem grupy $G$ na zbiorze $X$ oraz $x \in X$.
    Wówczas $\rho$ jest prymitywne wtedy i tylko wtedy, gdy $G_x$ jest maksymalną podgrupą $G$, tzn.
    nie istnieje podgrupa $H$ grupy $G$, taka że $G_x \lneq H \lneq G$.
\end{lemma}
\begin{proof}
Zauważmy najpierw, że warstwy (lewostronne) $G_x$ odpowiadają
jednoznacznie elementom zbioru $X$ -- bijekcja zadana jest wzorem
$\zeta \colon g G_x \mapsto x^g$. Funkcja ta jest dobrze określona
oraz jest iniekcją, gdyż $g_1 G_x = g_2 G_x \iff g_1^{-1} \cdot g_2
= h$ dla pewnego $h \in G_x$ $\iff x^{g_1^{-1} \cdot g_2} = x^h = x
\iff x^{g_1} = x^{g_2}$. Ponadto $\zeta$ jest surjekcją, gdyż
działanie jest tranzytywne. Stąd rzeczywiście $\zeta$ jest bijekcją.

Przejdźmy teraz do dalszej części dowodu.

$\Rightarrow)$ \\
Załóżmy nie wprost, że $G_x$ nie jest maksymalna, czyli istnieje
$H$, takie, że $G_x \lneq H \lneq G$. Skoro $H$ zawiera $G_x$, to
warstwy $H$ są sumami pewnych warstw $G_x$ -- jeżeli $g_1 G_x = g_2
G_x \iff g_1^{-1} \cdot g_2 \in G_x$ to również $g_1^{-1} \cdot g_2
\in H \iff g_1 H = g_2 H$. Stąd warstwy $H$ odpowiadają rozbiciu
zbioru warstw $G_x$, czyli również rozbiciu zbioru $X$. Zauważmy
jeszcze, że działanie $G$ zachowuje warstwy $H$. Jest tak dlatego,
że dla $g_1 H = g_2 H$ zachodzi $g_1^{-1} \cdot g_2 \in H$. Punkt
$g_i G_x = x^{g_i}$ przy działaniu elementem $f$ grupy $G$
przechodzi na $(x^{g_i})^f = x^{f g_i} = f g_i G_x$. Ale warstwy $f
g_1 G_x$ oraz $f g_2 G_x$ zawierają się w jednej warstwie $H$, gdyż
$(f g_1)^{-1} f g_2 = g_1^{-1} f^{-1} f g_2 = g_1^{-1} g_2 \in H$.

Otrzymaliśmy zatem system bloków, który na dodatek jest
nietrywialny, gdyż $H$ zawiera się ściśle pomiędzy $G_x$ a $G$.
Zatem działanie $\rho$ nie jest prymitywne -- sprzeczność. Stąd taka
grupa $H$ nie istnieje -- $G_x$ jest maksymalną podgrupą $G$.

$\Leftarrow)$ \\
Tutaj również przeprowadzimy dowód nie wprost. Załóżmy, że $\rho$
nie działa prymitywnie na $X$, czyli istnieje pewien nietrywialny
system bloków $\mathfrak{A}$. Niech $Y \in \mathfrak{A}$ będzie tym blokiem, który
zawiera $x$ oraz niech $H$ będzie stabilizatorem całego zbioru $Y$
(czyli zbiorem $\{g \in G \colon \forall_{y \in Y} y^g \in Y \}$).
Skoro $\mathfrak{A}$ jest nietrywialne, to $Y \ne X$ oraz istnieje
blok rozmiaru co najmniej 2. Ale z~poprzedniego lematu wiemy, że
wszystkie bloki mają tą samą wielkość, więc również $|Y| \ge 2$.

Zauważmy, że $H = \{ g \in G \colon x^g \in Y\} \stackrel{def}{=}
K$. Oczywiście $H \subseteq K$, gdyż elementy $H$ zachowują zbiór
$Y$. Z drugiej strony, jeżeli jakiś element z~$Y$ trafia z~powrotem
do $Y$, to całe $Y$ jest zachowywane, gdyż $Y$ jest elementem
systemu bloków. Stąd rzeczywiście $H = K$.

Na koniec wystarczy zobaczyć, że skoro $\{x\} \subsetneq  Y
\subsetneq X$, to $G_x \lneq H \lneq G$. Jest tak dlatego, że $H$, w
przeciwieństwie do $G_x$, zawiera elementy odwzorowujące $x$ na
jakiś inny element zbioru $Y$ ale nie zawiera elementów, które
odwzorowują $x$ na elementy spoza $Y$ (które istnieją). Stąd $G_x$
nie jest maksymalne -- sprzeczność. Stąd to działanie musi być
prymitywne.
\end{proof}

Teraz jesteśmy już gotowi na sformułowanie i dowód lematu Iwasawy.

\section{Lemat Iwasawy}

\begin{thh}
    Załóżmy, że $G$ jest skończoną grupą doskonałą, $\rho$ -- wiernie oraz prymitywnie działanie $G$ na zbiorze $X$.
    Załóżmy dodatkowo, że dla pewnego $x \in X$ stabilizator $G_x$ zawiera normalną podgrupę abelową $A$,
    której sprzężenia w $G$ generują całe $G$.
    Wówczas grupa $G$ jest prosta.
\end{thh}
\begin{proof}
    Załóżmy przeciwnie, że w $G$ istnieje właściwa, nietrywialna podgrupa normalna $K$.
    Skoro $G$ działa wiernie oraz $K$ jest nietrywialna, to $x_0^{k_0} \ne x_0$ dla pewnego $k_0 \in K$.
    Niech $H = G_{x_0}$.
    Dostajemy, że $K \nleq H$, gdyż $k_0 \not \in H$, stąd również $H \lneq H K$.

    Z lematu \ref{max_group} otrzymujemy, że $H$ jest podgrupą maksymalną w $G$.
    $H \lneq H K$, więc $H K = G$.
    Stąd (i z~twierdzenia \ref{mult_groups}) każdy element $g \in G$ jest postaci $g = h k$, gdzie $h \in H$ oraz $k \in K$.

    Skoro działanie $\rho$ jest prymitywne, a więc tranzytywne, to z~twierdzenia \ref{conj_stab}
    dostajemy, że $G_x$ jest sprzężone z~$H$.
    Z założenia dodatkowo wynika, że $H$ zawiera podgrupę $B$ sprzężoną do $A$,
    ponadto $B$ jest normalną podgrupą abelową $H$, której sprzężenia (w $G$) generują całe $G$.
    Sprzężenia $B$ są postaci $g^{-1} B g = k^{-1} h^{-1} B h k = k^{-1} B k \le B K$.
    Wszystkie sprzężenia $B$ generują $G$ i są zawarte w $B K \le G$, stąd $G = B K$.

    Korzystając z~trzeciego twierdzenia o izomorfizmie dostajemy:
    $$ G/K = BK/K \simeq B/B\cap K$$
    Ale grupa ilorazowa grupy abelowej jest abelowa, stąd zarówno $B/B\cap K$ jaki i $G/K$ są abelowe.
    Z twierdzenia o komutancie wnioskujemy, że $K \ge [G, G] = G$, gdyż $G$ jest grupą doskonałą --
    dostaliśmy sprzeczność z~założeniem, że $K$ jest właściwą podgrupą $G$, stąd $G$ jest grupą prostą.
\end{proof}

Zastosowania lematu Iwasawy znajdują się w dalszej części pracy.



\chapter{Prostota specjalnej rzutowej grupy liniowej $PSL_n(k)$}

%\chapter{Prostota $A_n$ z~lematu Iwasawy}

\begin{thebibliography}{99}
\addcontentsline{toc}{chapter}{Bibliografia}

\bibitem[Wil09]{Simple_groups} Robert A. Wilson, \textit{The Finite Simple Groups}, Springer, 2009.
\bibitem[Bia87]{BB} Andrzej Białynicki-Birula, \textit{Zarys algebry}, Państwowe Wydawnictwo Naukowe, 1987.
\bibitem[Lan73]{Lang} Serge Lang, \textit{Algebra}, Państwowe Wydawnictwo Naukowe, 1973.
\bibitem[Kar76]{Kar_Mierz} M. I. Kargapołow, J. I. Mierzlakow, \textit{Podstawy teorii grup}, Państwowe Wydawnictwo Naukowe, 1976.
\bibitem[Bag02]{Baginski} Czesław Bagiński, \textit{Wstęp do teorii grup}, Script, 2002.
\bibitem[Neu03]{G_and_G} Peter M. Neumann, Gabrielle A. Stoy, Edward C. Thompson, \textit{Groups and Geometry}, Oxford Science Publications, 2003.

\end{thebibliography}

\end{document}

% \cite{B}