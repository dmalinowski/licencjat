\documentclass[licencjacka]{pracamgr}
\usepackage{polski}

\usepackage[utf8]{inputenc}
%\usepackage[cp1250]{inputenc}

\author{Daniel Malinowski}
\nralbumu{292680}

\title{Metody dowodzenia prostoty grup}

\tytulang{Methods of proving the simplicity of groups}

\kierunek{Matematyka}

\opiekun{dra hab. Zbigniewa Marciniaka\\
  Instytut Matematyki\\}

\date{Czerwiec 2013}

\dziedzina{ 
11.1 Matematyka\\ 
}

%Klasyfikacja tematyczna wedlug AMS (matematyka) lub ACM (informatyka)
%\klasyfikacja{D. Software\\
%  D.127. Blabalgorithms\\
%  D.127.6. Numerical blabalysis}
\klasyfikacja{20. Group theory and generalizations}

\keywords{grupa prosta, grupa alternująca, grupa specjalna rzutowa liniowa, lemat Iwasawy}

% Tu jest dobre miejsce na Twoje w?asne makra i~?rodowiska:
\newtheorem{deff}{Definicja}[section]
\newtheorem{thh}{Twierdzenie}[section]
\newtheorem{lemma}{Lemat}[section]
% koniec definicji

\begin{document}
\maketitle

\begin{abstract}
  % TODO
\end{abstract}

\tableofcontents
%\listoffigures
%\listoftables


\chapter*{Wprowadzenie}
\addcontentsline{toc}{chapter}{Wprowadzenie}

\chapter{Wiadomości wstępne}

\chapter{Prostota grupy alternującej $A_n$}

\chapter{Lemat Iwasawy}

\chapter{Prostota specjalnej rzutowej grupy liniowej $PSL_n(k)$}

%\chapter{Prostota $A_n$ z lematu Iwasawy}

\begin{thebibliography}{99}
\addcontentsline{toc}{chapter}{Bibliografia}

\bibitem[A]{B} C, \textit{D}, E


\end{thebibliography}

\end{document}


% \cite{B}

%\begin{def}\label{Co}
%\end{def}

%\chapter{}
%	\section{}
%\appendix
%\chapter{} % - zrobi dodatek