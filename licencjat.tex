\documentclass[licencjacka]{pracamgr}
\usepackage{polski}

\usepackage[utf8]{inputenc}
\usepackage{amssymb}
\usepackage{enumitem}
\usepackage{eufrak}
%\usepackage[cp1250]{inputenc}

\author{Daniel Malinowski}
\nralbumu{292680}

\title{Metody dowodzenia prostoty grup}

\tytulang{Methods of proving the simplicity of groups}

\kierunek{Matematyka}

\opiekun{dra hab. Zbigniewa Marciniaka\\
  Instytut Matematyki\\}

\date{Czerwiec 2013}

\dziedzina{ 
11.1 Matematyka\\ 
}

%Klasyfikacja tematyczna wedlug AMS (matematyka) lub ACM (informatyka)
%\klasyfikacja{D. Software\\
%  D.127. Blabalgorithms\\
%  D.127.6. Numerical blabalysis}
\klasyfikacja{20. Group theory and generalizations}

\keywords{grupa prosta, grupa alternująca, grupa specjalna rzutowa liniowa, lemat Iwasawy}

% Tu jest dobre miejsce na Twoje w?asne makra i~?rodowiska:
\newtheorem{deff}{Definicja}[section]
\newtheorem{thh}{Twierdzenie}[section]
\newtheorem{lemma}{Lemat}[section]
\newtheorem{fact}{Fakt}[section]
% koniec definicji

\begin{document}
\maketitle

\begin{abstract}
  % TODO
\end{abstract}

\tableofcontents
%\listoffigures
%\listoftables


\chapter*{Wprowadzenie}
\addcontentsline{toc}{chapter}{Wprowadzenie}

\chapter{Wiadomości wstępne}

Rozdział ten zawiera przypomnienie pewnych definicji, własności i twierdzeń omawianych na podstawowym kursie algebry I
oraz ustalenie oznaczeń.

W niniejszej pracy dużymi literami alfabetu (np. $G, H, K$) będą oznaczane grupy.
Ich elementy będą oznaczanie małymi literami alfabetu (np. $g, h, k$), przy czym przez $e$ będzie zawsze oznaczany element neutralny.
Rozważane grupy będą (w większości) nieprzemienne, w związku z tym będzie stosowany zapis multiplikatywny.

\section{Grupy proste}

Zacznijmy zatem od przypomnienia podstawowej definicji w tej pracy.
\begin{deff}
	Nietrywialną grupę $G$ nazwiemy \emph{grupą prostą}, jeżeli nie ma ona podgrup normalnych różnych od $\{e\}$ oraz samej siebie.
\end{deff}
\begin{fact}
	Jedynymi (z dokładnością do izomorfizmu) przemiennymi grupami prostymi są skończone grupy cykliczne o liczbie elementów będącą liczbą pierwszą.
\end{fact}
Jest to prosta konsekwencja tego, że w grupach przemiennych wszystkie podgrupy są podgrupami normalnymi.
% jak zrobić odwołanie po szczegóły?


\section{Twierdzenia o izomorfizmie}
Przejdźmy teraz do podstawowych twierdzeń o izomorfizmie.
\begin{thh}[Pierwsze twierdzenie o izomorfizmie] $ $ \\
	Niech $G, H$ -- grupy, 	$ \varphi \colon G \to H$ homomorfizm, $K = \ker{\varphi}$ oraz $H' = \mathrm{im}\varphi $. \\
	Wówczas zachodzi izomorfizm $$G/K \simeq H'$$
\end{thh}
\begin{thh}[Drugie twierdzenie o izomorfizmie]$ $\\
	Niech $G$ -- grupa, 	$H_1, H_2$ podgrupy normalne $G$, przy czym $H_2 \leq H_1$. \\
	Wówczas $H_2 \trianglelefteq H_2$, $H_1/H_2 \trianglelefteq G/H_2$ i zachodzi izomorfizm
	$$ (G/H_2) / (H_1/H_2) \simeq G/H_1$$
\end{thh}
\begin{thh}[Trzecie twierdzenie o izomorfizmie] $ $ \\
	Niech $G$ -- grupa, $H_1$ podgrupa normalna $G$, $H$ podgrupa $H_1$. \\
	Wówczas $H\cap H_1 \trianglelefteq H$ oraz zachodzi izomorfizm
	$$ H/(H\cap H_1) \simeq H \cdot H_1 / H_1 $$
\end{thh}


\section{Komutant i abelianizacja}
Poniżej przedstawionych jest kilka użytecznych wiadomości o komutancie.
\begin{deff}
	Niech $G$ będzie dowolną grupą. Wówczas \emph{komutantem grupy $G$} nazywamy podgrupę $G$ generowaną przez wszystkie elementy postacji
	$aba^{-1}b^{-1}$, gdzie $a, b \in G$. Komutant grupy $G$ oznaczamy przez $[G, G]$.
\end{deff}
\begin{thh}[O komutancie] $ $\\
	Komutant $[G,G]$ jest podgrupą normalną $G$, przy czym grupa ilorazowa $G/[G,G]$ jest grupą abelową.
	Ponadto dla dowolnej podgrupy normalnej $H \trianglelefteq G$ takiej, że $G/H$ jest abelowa, zachodzi $[G,G] \le H$.
\end{thh}
\begin{deff}
	Przekształcenie kanoniczne $G \to G/[G,G]$ (rzutowanie na grupę ilorazową) nazywamy abelianizacją.
\end{deff}

O abelianizacji (w przeciwieństwie do twierdzenia o komutancie) nie będzie więcej wspominane w tej pracy, 
ale ta definicja została przytoczona w celu domknięcia podstawowych faktów o komutancie.
Ważniejszym dla nas pojęciem jest pojęcie grupy doskonałej:

\begin{deff}
	\emph{Grupą doskonałą} nazwiemy dowolną grupę, która jest równa swojemu komutantowi.
\end{deff}

Grupami doskonałymi zajmiemy się w dalszej części pracy -- przy lemacie Iwasawy.
Na razie zauważmy prosty fakt:

\begin{fact}
	Nieprzemienne grupy proste są grupami doskonałymi.
\end{fact}


\section{Działanie grupy na zbiorze}
Na koniec tego rozdziału przyjrzyjmy się jednej z ważniejszej własności grup -- ich możliwości działania na zbiorach.

\begin{deff}
 	Niech $G$ będzie grupą, a $X$ -- zbiorem. Mówimy, że \emph{$\rho$ jest działaniem grupy $G$ na zbiorze $X$}, 
	jeżeli dla każdego $g \in G$ przyporządkowane jest przekształcenie $\rho_g\colon X \to X$, takie, że:
	\begin{itemize}
		\item $\rho_e = \mathrm{id}_X$,
		\item $\rho_g \circ \rho_h = \rho(gh)$, dla dowolnych $g, h \in G$.
	\end{itemize}
	Jeżeli sposób działania ($\rho$) wynika z kontekstu, to zamiast $\rho_g(x)$ będziemy pisać $x^g$.
\end{deff}

Zgrabniejszy opis działania grupy na zbiorze daje poniższe twierdzenie.
Zanim jednak do niego przejdziemy, przypomnijmy sobie jeszcze jedną definicję.

\begin{deff}
	Niech $X$ będzie dowolnym zbiorem. Wówczas \emph{grupą symetrii zbioru X} nazywamy zbiór bijekcji $X \to X$,
	wraz z operacją składania. Grupę tę oznaczamy $S_X$.
\end{deff}

\begin{thh}[O działaniu grupy na zbiorze] $ $ \\
	Niech $G$ będzie grupą, a $X$ -- zbiorem. Wówczas $\rho$ jest działaniem $G$ na $X$ wtedy i tylko wtedy, 
	gdy $\rho$ jest homomorfizmem z $G$ w grupę symetrii zbioru $X$.
\end{thh}

Z działaniem grupy na zbiorze związane jest dużo ważnych definicji i twierdzeń.
Poniżej przytoczone są te najistotniejsze z punktu widzenia tej pracy.

\begin{deff}
	Załóżmy, że $\rho$ jest działaniem grupy $G$ na zbiorze $X$ oraz $x \in X$. Wówczas:
	\begin{enumerate}[label=\alph*)]
	 \item \emph{Stabilizatorem punktu $x$ (grupą izotropii $x$)} nazwiemy zbiór elementów $\{g \in G\colon x^g = x \}$. 
					Stabilizator punktu $x$ oznaczamy $G_x$.
	 \item \emph{Orbitą punktu $x$} nazwiemy podzbiór $X$ równy $\{y \in X \colon \exists_{g \in G} x^g = y \}$.
					Orbitę punktu $x$ oznaczamy $G(x)$.
	\end{enumerate}
% punkt stały ?
\end{deff}

Podstawowe własności tych obiektów przedstawia następujący fakt:
\begin{fact}
	Załóżmy, że $\rho$ jest działaniem grupy $G$ na zbiorze $X$ oraz $x, y \in X$. Wówczas:
	\begin{enumerate}[label=\alph*)]
	 \item $G_x$ jest podgrupą $G$.
	 \item $G(x)$ i $G(y)$ są równe lub rozłączne (orbity tworzą rozbicie zbioru $X$).
	\end{enumerate}
\end{fact}

Zanim przejdziemy do ważniejszych twierdzeń opisujących orbity i stabilizatory, 
przypomnijmy wcześniej, jakie własności może mieć działanie grupy na zbiorze.

\begin{deff}
	Załóżmy, że $\rho$ jest działaniem grupy $G$ na zbiorze $X$.
	\begin{enumerate}[label=\alph*)]
	 \item \emph{$\rho$ jest działaniem tranzytywnym (przechodnim)}, jeżeli wszystkie elementy $X$ tworzą jedną orbitę.
	 \item \emph{$\rho$ jest działaniem wiernym}, jeżeli $\rho$ jest iniekcją jako homomorfizm $G \to S_X$.
	\end{enumerate}
\end{deff}

Jak to zostało wcześniej zapowiedziane, na koniec przytoczmy dwa ważne twierdzenie pokazujące zależność między orbitami a stabilizatorami.

\begin{thh}[O orbitach i stabilizatorach] $ $\\
	Załóżmy, że $\rho$ jest działaniem grupy $G$ na zbiorze $X$, przy czym $X$ jest zbiorem skończonym.
	Ponadto $x \in X$. Wówczas $|G(x)| = [G : G_x]$.
\end{thh}

\begin{thh}[Równanie klas] $ $\\
	Przy założeniach z poprzedniego twierdzenia zachodzi
	$$ |X| = \sum_{i=1}^k [G : G_{x_i}] ,$$
	gdzie $x_1, x_2, \cdots, x_k$ to reprezentanci wszystkich orbit działania $\rho$.
\end{thh}



\chapter{Prostota grupy alternującej $A_n$}




\chapter{Lemat Iwasawy}
W tym rozdziale przedstawione zostanie jedno z ważniejszych narzędzi do dowodzenia prostoty grup -- lemat Iwasawy.
Lecz najpierw wprowadzimy nowe pojęcie --  prymitywność.

\section{Prymitywne działania grupy}
Jak zostało to już wspomniane w wiadomościach wstępnych, działanie grupy $G$ na zbiorze $X$ jest tranzytywne, 
jeżeli elementy $X$ tworzą jedną orbitę, czyli dla dowolnych $x, y \in X$ istnieje $g \in G$ takie, że $x^g = y$.
Teraz uogólnimy to pojęcie.

\begin{deff}
	Załóżmy, że $\rho$ jest działaniem grupy $G$ na zbiorze $X$. \\
	\emph{$\rho$ jest działaniem $k$-tranzytywnym ($k$-przechodnim)}, 
	jeżeli dla dowolnych ciągów $k$ elementowych $(a_1, a_2, \cdots, a_k)$ oraz $(b_1, b_2, \cdots, b_k)$, 
	które składają się z różnych elementów z $X$ istnieje taki element $g$ z grupy $G$, że
	$a_i^g = b_i^g$ dla każdego $i = 1, 2, \cdots, k$.
\end{deff}
W szczególności $1$-tranzytywność to jest dokładnie to samo, co zwykła tranzytywność.

Aby lepiej zilustrować to pojęcie, policzmy ilu tranzytywne jest naturalne działanie grupy $S_n$ oraz $A_n$ na zbiorze $X = \{1, 2, \cdots, n\}$,
tzn. takie, w którym $i^\sigma = \sigma(i)$.

Jak łatwo zauważyć, działanie $S_n$ jest $n$-tranzytywne -- skoro $S_n$ składa się ze wszystkich permutacji, 
to zawsze możemy odwzorować ciąg $(a_1, a_2, \cdots, a_n)$ na $(b_1, b_2, \cdots, b_n)$, gdyż jak założyliśmy w definicji, 
wszystkie $a_i$ jaki i wszystkie $b_i$ są parami różne.
Stąd również działanie $S_n$ jest $k$-tranzytywne dla każdego $k \le n$.

Natomiast w $A_n$ nie ma wszystkich permutacji, zatem działanie $A_n$ nie może być $n$-tranzytywne.
Nie może być również $(n-1)$-tranzytywne, gdyż skoro mówimy na co przechodzą $n-1$ elementy $X$ i ma to być permutacja,
to wartość ostatniego elementu też jest ustalona, czyli wybór $(n-1)$ pozycji jest tak na prawdę wyborem wszystkich $n$ pozycji, 
a na wszystkich elementach nie możemy dowolnie ustalić permutacji.
Zauważmy jednak, że działanie $A_n$ jest $(n-2)$-tranzytywne. 
Rzeczywiście, chcąc żeby $a_i$ przeszło na $b_i$ dla $i = 1, 2, \cdots, (n-2)$ mamy do wyboru dwie permutacje.
Jedna z nich odwzorowuje $x \mapsto y, x' \mapsto y'$, a druga $x \mapsto y', x' \mapsto y$, 
gdzie $x, x'$ to elementy nie wybrane na $a_i$, a $y, y'$ to elementy nie wybrane na $b_i$. 
Ale te permutacje różnią się o transpozycję $(y, y')$, zatem jedna z nich jest parzysta, czyli należy do $A_n$,
więc rzeczywiście możemy odwzorować $(a_1, a_2, \cdots, a_{n-2})$ na $(b_1, b_2, \cdots, b_{n-2})$.
Stąd działanie $S_n$ jest $k$-tranzytywne dla każdego $k \le n-2$.

Wprowadzimy teraz własność prymitywności. 
Będzie to coś pomiędzy tranzytywnością a $2$-tranzytywnością.

\begin{deff}
	Załóżmy, że $\rho$ jest działaniem grupy $G$ na zbiorze $X$. \\
	\emph{Systemem bloków} działania $\rho$ nazywamy podział zbioru $X$ zachowywany przez $\rho$,
	tzn. rodzinę zbiorów $\mathfrak{A} = \{Y_i \colon i \in I \}$, które są niepuste, parami rozłączne, sumują się do $X$ 
	oraz dla dowolnych $Y \in \mathfrak{A}, x, x' \in Y$ oraz $g \in G$ 
	oba elementy $x^g$ oraz $x'^g$ znajdują się razem w jednym zbiorze $Y' \in \mathfrak{A}$.
\end{deff}

Zauważmy, że zawsze mamy co najmniej dwa systemy bloków -- jeden blok z całym zbiorem $\mathfrak{A} = \{X\}$
oraz system z wszystkimi blokami jednoelementowymi $\mathfrak{A} = \{\{x\} \colon x \in X\}$.
W związku z tym naturalna jest definicja:

\begin{deff}
	\emph{Nietrywialnym systemem bloków} nazywamy dowolny system bloków, 
	który jest różny od dwóch wyżej wspomnianych -- z jednym blokiem lub z blokami jednoelementowymi.
\end{deff}


\chapter{Prostota specjalnej rzutowej grupy liniowej $PSL_n(k)$}

%\chapter{Prostota $A_n$ z lematu Iwasawy}

\begin{thebibliography}{99}
\addcontentsline{toc}{chapter}{Bibliografia}

\bibitem[A]{B} C, \textit{D}, E


\end{thebibliography}

\end{document}


% \cite{B}

%\begin{def}\label{Co}
%\end{def}

%\chapter{}
%	\section{}
%\appendix
%\chapter{} % - zrobi dodatek